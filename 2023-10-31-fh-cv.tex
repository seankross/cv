% CV template for Sean Kross

% Derived from http://links.tedpavlic.com/tex/tpavlic_cv_faculty.tex
\input{tedpavlic-template}
\usepackage[utf8]{inputenc}

%\def\public{XXX} % define for a public version of my CV

\usepackage{etoolbox}
\usepackage{comment}

\begin{document}
\makeheading{Sean Kross}

\section{}

\hyphenation{Micro-soft}

% NOTE: Mind where the & separators and \\ breaks are in the following
%       table. Table is one row made up of three parboxes. The left
%       parbox has address info, the middle parbox has a vertical bar,
%       and the right parbox has phone and electronic contact
%       information.
%
% MACROS: \rcollength is the width of the right column of the table
%             (adjust it to your liking; default is 1.85in).
\newlength{\rcollength}\setlength{\rcollength}{1.85in}%
%
\vspace{1em}

\begin{tabular}[t]{@{}p{\textwidth-\rcollength}p{\rcollength}}%

\parbox{\textwidth-\rcollength}{Data Staff Scientist \\
Data Science Lab (DaSL) \\
Fred Hutchinson Cancer Research Center
}

&

% Non-snail-mail contact information
\parbox{\rcollength}{

\ifx\public\undefined
%Phone: 000.000.0000 \\
Updated: \today \\
\else
Updated: \today \\
\fi
Email: sean@seankross.com \\
www.seankross.com
}

\end{tabular}

\vspace{1.75em}

\section{}

\textbf{RESEARCH INTERESTS}
\vspace{0.05in}

Human-computer interaction, data science, online learning, cybersecurity

\section{} \vspace{0.125in} \textbf{ACADEMIC POSITIONS} \vspace{-0.5em}

\section{07/2022 --}
\textbf{The Fred Hutchinson Cancer Research Center}, Seattle, WA \\
Data Staff Scientist

\section{01/2018 --}
\textbf{The Johns Hopkins Bloomberg School of Public Health}, Baltimore, MD \\
Affiliate Faculty of Biostatistics

\section{01/2017 -- 05/2017}
\textbf{The University of Pennsylvania School of Design}, Philadelphia, PA \\
Adjunct Faculty of Design

\section{} \vspace{0.1in} \textbf{EDUCATION} \vspace{-0.5em}

\section{09/2017 -- 06/2022}

\textbf{The University of California, San Diego}, La Jolla, CA \\
Ph.D.\ in Cognitive Science \\
%Dissertation: \emph{Software Tools to Facilitate Research Programming} \\
Advisor: Philip Guo %, Jeffrey Heer, Margo Seltzer (Harvard University)

\section{09/2017 -- 03/2020}

\textbf{The University of California, San Diego}, La Jolla, CA \\
M.S.\ in Cognitive Science \\
Advisor: Philip Guo

\section{06/2012 -- 05/2015}

\textbf{The University of Maryland, College Park}, College Park, MD \\ Bachelor
of Science in Computer Science

\section{09/2008 -- 05/2012}

\textbf{New York University}, New York, NY \\
Bachelor of Arts in Biology

\section{} \vspace{0.1in} \textbf{AWARDS AND HONORS} \vspace{-0.05in}

\section{08/2020} USENIX Security Distinguished Paper Award [C.10]

\section{05/2019} CHI Honorable Mention Paper Award [C.7]

\section{05/2012} New York University President's Service Award

%\section{09/2009 -- 06/2011}
%National Science Foundation (NSF) Graduate Fellowship

\section{} \vspace{0.1in} \textbf{PRIOR EMPLOYMENT} \vspace{-0.5em}

\section{01/2020 -- 06/2022}
\textbf{Logan Circle Labs}, Seattle, WA \\
Lead Consultant \\
- Managed a team of 3 subcontractors on software projects.

\section{01/2022 -- 06/2022}
\textbf{UC San Diego}, San Diego, CA \\
\textit{Instructor of Record} \\
- Managed a team of 5 graduate and undergraduate students. \\
- Led a classroom of 200 students.

\section{01/2020 -- 12/2021}
\textbf{UC San Diego}, San Diego, CA \\
\textit{Graduate Research Assistant} \\
- Managed a team of 2 junior graduate students. \\
- Led academic research projects.

\section{05/2021 -- 08/2021}
\textbf{Microsoft Research}, Seattle, WA \\
Research Intern -- Research in Software Engineering (RiSE) Group \\
Advised by Thomas Ball

\section{05/2020 -- 08/2020}
\textbf{Microsoft Research}, New York, NY \\
Research Intern -- Computational Social Science Group \\
Advised by Jake Hofman \& Daniel Goldstein

\section{08/2017 -- 12/2019}
\textbf{UC San Diego}, San Diego, CA \\
\textit{Graduate Teaching Assistant} \\
- Managed a team of 2 junior graduate students. \\
- Led academic research projects.

\section{01/2017 -- 06/2017}
\textbf{The University of Pennsylvania}, Philadelphia, PA \\
\textit{Instructor of Record} \\
- Managed a team of 2 graduate students. \\
- Led a classroom of 30 students.

\section{09/2015 -- 09/2017}
\textbf{The Johns Hopkins Bloomberg School of Public Health}, Baltimore, MD \\
Chief Technology Officer -- Johns Hopkins Data Science Lab

\section{01/2014 -- 07/2015}
\textbf{University of Maryland}, College Park, MD \\
Research Assistant --  Computational Biology Group -- Advisor: Mihai Pop

\section{01/2012 -- 05/2012}
\textbf{New York University}, New York, NY \\
Research Assistant --  Genomics Group -- Advisor: Patrick Eichenberger

% \newpage % stent
%
% \section{} \vspace{0.2in} \textbf{FUNDING}
%
%\section{} National Science Foundation. NRT-IGE: Augmenting, Piloting,
%and Scaling Computational Notebooks to Train New Graduate Researchers in
%Data-Centric Programming. \$498,751 (co-PI, 2017--2020. PI: James
%Hollan, co-PI: Philip Guo, co-PI: Scott Klemmer, co-PI: Bradley Voytek)
%

\section{} \vspace{0.2in} \textbf{PUBLICATIONS} \vspace{-0.05in}

%Note that in many areas within computer science and human-computer
%interaction, \emph{conferences} (not journals) are the primary venues
%for peer-reviewed publications.

\section{Conference \\ Papers}

\begin{bibenum}

\item[C.15] Sam Lau*, \textbf{Sean Kross*}, Eugene Wu, Philip J.\ Guo  (*equal contribution). 
Teaching Data Science by Visualizing Data Table Transformations: Pandas Tutor for Python, Tidy Data Tutor for R, and SQL Tutor.
In Proceedings of DataEd 2023: \emph{ACM International Workshop on Data Systems Education}, June 2023. Seattle, WA, USA.

\item[C.14] \textbf{Sean Kross} and Philip J.\ Guo. 
Five Pedagogical Principles of a User-Centered Design Course that Prepares Computing Undergraduates for Industry Jobs.
In Proceedings of SIGCSE 2022: \emph{ACM Technical Symposium on Computer Science Education}, March 2022. Providence, RI, USA.

\item[C.13] \textbf{Sean Kross} and Philip J.\ Guo. 
Orienting, Framing, Bridging, Magic, and Counseling: How Data Scientists Navigate the Outer Loop of Client Collaborations in Industry and Academia.
In Proceedings of CSCW 2021: \emph{ACM Conference on Computer Supported Cooperative Work}, November 2021. Toronto, Canada.

\item[C.12] \textbf{Sean Kross}, Eszter Hargittai, Elissa M.\ Redmiles. 
Characterizing the Online Learning Landscape: What and How People Learn Online.
In Proceedings of CSCW 2021: \emph{ACM Conference on Computer Supported Cooperative Work}, November 2021. Toronto, Canada.

\item[C.11] Xiaoying Pu, \textbf{Sean Kross}, Jake Hofman, Daniel Goldstein. 
Datamations: Animated Explanations of Data Analysis Pipelines.
In Proceedings of CHI 2021: \emph{ACM Conference on Human Factors in Computing 
Systems}, May 2021. Yokohama, Japan.

\item[C.10] Elissa M.\ Redmiles, Noel Warford, Amritha Jayanti, Aravind Koneru,
\textbf{Sean Kross}, Miraida Morales, Rock Stevens, Michelle L.\ Mazurek.
A Comprehensive Quality Evaluation of Security and Privacy Advice on the Web. 
In Proceedings of the 2020 USENIX Security Symposium. August 2020.
\\ ({\textbf{Distinguished Paper Award}})

\item[C.9] Elissa M.\ Redmiles, Lisa Maszkiewicz, Emily Hwang, Dhruv Kuchhal,
Everest Liu, Miraida Morales, Denis Peskov, Sudha Rao, Rock Stevens, 
Kristina Gligorić, \textbf{Sean Kross}, Michelle L.\ Mazurek, Hal Daumé III.
Comparing and Developing Tools to Measure the Readability of Domain-Specific 
Texts. In Proceedings of EMNLP 2019: \emph{ACL SIGDAT Conference on Empirical 
Methods in Natural Language Processing}, November 2019. Hong Kong, China.

\item[C.8] \textbf{Sean Kross} and Philip J.\ Guo. End-User Programmers 
Repurposing End-User Programming Tools to Foster Diversity in Adult End-User 
Programming Education. In Proceedings of VL/HCC 2019: \emph{IEEE
Symposium on Visual Languages and Human-Centric Computing}, October 2019. 
Memphis, TN, USA.

%\vspace{0.1in}

\item[C.7] \textbf{Sean Kross} and Philip J.\ Guo. Practitioners Teaching Data 
Science in Industry and Academia: Expectations, Workflows, and Challenges.
In Proceedings of CHI 2019: \emph{ACM Conference on Human Factors in Computing 
Systems}, May 2019. Glasgow, UK. \\ ({\textbf{Honorable Mention Paper Award}})

\item[C.6] Elissa M.\ Redmiles, \textbf{Sean Kross}, Michelle L.\ Mazurek. 
How Well Do My Results Generalize? Comparing Security and Privacy Survey Results
from MTurk, Web, and Telephone Samples. In Proceedings of S\&P 2019: 
\emph{IEEE Symposium on Security and Privacy}, May 2019. San Francisco, CA, USA.

\item[C.5] Elissa M.\ Redmiles, Ziyun Zhu, \textbf{Sean Kross}, Dhruv Kuchhal,
Tudor Dumitras, Michelle L.\ Mazurek. Asking for a Friend: Evaluating Response 
Biases in Security User Studies. In Proceedings of CCS 2018: \emph{ACM 
Conference on Computer and Communications Security}, October 2018. Toronto, ON, 
Canada.

\item[C.4] \textbf{Sean Kross} and Philip J.\ Guo. Students, Systems,
and Interactions: Synthesizing the First Four Years of Learning@Scale
and Charting the Future. In Proceedings of L@S 2018: \emph{ACM
Conference on Learning at Scale}, June 2018. London, UK.

\item[C.3] Elissa M.\ Redmiles, \textbf{Sean Kross}, Michelle L.\ Mazurek. 
Where is the Digital Divide? Examining the Impact of Socioeconomics on Security 
and Privacy Outcomes. In Proceedings of CHI 2017: \emph{ACM Conference on 
Human Factors in Computing Systems}, May 2017. Denver, CO, USA.

\item[C.2] Elissa M.\ Redmiles, \textbf{Sean Kross}, Michelle L.\ Mazurek. How 
I Learned to be Secure: a Census-Representative Survey of Security Advice 
Sources and Behavior. In Proceedings of CCS 2016: \emph{ACM Conference on 
Computer and Communications Security}, October 2016. Vienna, Austria.

\item[C.1] Elissa Redmiles, Mary Allison Abad, Isabella Coronado, 
\textbf{Sean Kross}, Amelia Malone. A Classroom Tested Accessible Multimedia 
Resource for Engaging Underrepresented Students in Computing. In Proceedings of
SIGCSE 2015: \emph{ACM Conference on Computer Science Education}, March 2015. 
Kansas City, MO, USA.

\end{bibenum}

\vspace{0.1in}

%\section{Preprints and Technical Reports}

%\begin{bibenum}

%\item[P.1] \textbf{Sean Kross}, Roger D.\ Peng, Brian S.\ Caffo, Ira Gooding,
%Jeffery T.\ Leek. The Democratization of Data Science Education. 
%In \emph{PeerJ Preprints}, August 2017.

%\end{bibenum}

\section{Journal \\ Papers}

\begin{bibenum}

\item[J.4] Emily J. Winokur, Cherry Song, Estelita S. Leija, Joanne Chen, 
\textbf{Sean Kross}, Danielle Shamam, Marcelo Aguilar-Rivera, Laleh K. Quinn, 
Federico Rossano, Andrea A. Chiba.
Reciprocity in Dyads and Triads: Female Rats Alter Their Prosocial Behavior 
According to the Social Context.
In \emph{Animal Behavior and Cognition}, 2023.

\item[J.3] \textbf{Sean Kross}, Jeffery T.\ Leek, John Muschelli.
Ari: The Automated R Instructor. 
In \emph{The R Journal}, 2020.

\item[J.2] Lucy D'Agostino McGowan, \textbf{Sean Kross}, Jeffery T.\ Leek.
Tools for Analyzing R Code the Tidy Way. 
In \emph{The R Journal}, 2020.

\item[J.1] \textbf{Sean Kross}, Roger D.\ Peng, Brian S.\ Caffo, Ira Gooding,
Jeffery T.\ Leek. The Democratization of Data Science Education. 
In \emph{The American Statistician}, 2020.

\end{bibenum}

\vspace{0.1in}

\section{Books}

\begin{bibenum}

\item[B.2] \textbf{Sean Kross}. The Unix Workbench. Leanpub, 2017.

\item[B.1] Roger D.\ Peng, \textbf{Sean Kross}, Brooke Anderson. Mastering 
Software Development in R. Leanpub, 2017.

\end{bibenum}

%\vspace{0.1in}

\section{} \vspace{0.2in} \textbf{FUNDING} \vspace{0.05in}

\begin{bibenum}

\item[F.2] Alfred P. Sloan Foundation. ``Investigating How Domain Scientists and Computational Scientists Work Together in Software-driven Research Labs" \$49,681. PI: Philip Guo, co-PI: \textbf{Sean Kross}. 2021-2022.

\item[F.1] Facebook Research: Economic Opportunity and Digital Platforms. ``Learning Online: Who Benefits, Who is Left Behind?" \$50,000. PI: Eszter Hargittai, co-PI: \textbf{Sean Kross}, co-PI: Elissa M. Redmiles. 2020.

\end{bibenum}

\vspace{0.1in}

\section{} \textbf{TALKS} \vspace{0.05in}

\begin{innerlist}
\item Visualizing Data Analysis Pipelines with Pandas Tutor and Tidy Data Tutor. \textit{Posit Conf}, September 2023.
\item How Data Scientists Navigate the Outer Loop of Client Collaborations in Industry and Academia. \textit{Joint Statistical Meetings}, August 2021.
\item Create a Personal Website in 5 Minutes with Postcards. \textit{Cascadia R Conference}, June 2021.
\item The Challenges of Analytic Workflows: Perspectives from Data Science Educators. \textit{The Eastern North American Region of the International Biometric Society (ENAR)}, Nashville, TN, March 2020.
\item The Lean Course: Open and Collaborative Online Course Development. \textit{Joint Statistical Meetings}, Vancouver, BC, Canada, July 2018.
\item Lessons from Teaching Data Science to Over a Million People. \textit{Crunch Conference}, Budapest, Hungary, October 2017.
\item Beyond Axes: Simulating Systems with Interactive Graphics. \textit{Joint Statistical Meetings}, Baltimore, MD, August 2017.
\item Re-thinking the Value Added by a Data Repository. \textit{Dataverse Community Meeting}, Cambridge, MA, June 2017.
\item Scaling Data Science Education. \textit{Seminário Internacional de Estatística}, Rio de Janeiro, Brazil, May 2016.
\end{innerlist}

%\newpage % stent

\section{} \textbf{TEACHING} \vspace{-0.05in}

\section{Instructor}

\begin{innerlist}
\item UCSD COGS 127: Data-Driven User Experience and Product Design  \\ (Winter 2022)
\item Johns Hopkins University on Coursera.org: The Unix Workbench \\ (Continually since Summer 2017)
\item The University of Pennsylvania MUSA 850: Data Wrangling and \\ Data Visualization (Spring 2017)
\end{innerlist}

\section{Teaching\\Assistant}

\begin{innerlist}
\item UCSD COGS 127: Data-Driven User Experience and Product Design  \\ (Fall 2019, 2020)
\item UCSD COGS 9: Introduction to Data Science  \\ (Winter 2019)
\item UCSD COGS 120/CSE 170: Human-Computer Interaction Design \\ (Fall 2017, 2018)
\item UCSD COGS 121 Human-Computer Interaction Programming Studio \\ (Spring 2018)
\end{innerlist}

\begin{comment}

\section{} \vspace{0.2in} \textbf{SERVICE} \vspace{0.05in}

\section{External Paper\\Reviewer}

The Journal of Open Source Software (2018), The American Statistician (2017)

\section{} \vspace{0.2in} \textbf{SOFTWARE} \vspace{0.05in}

\begin{innerlist}
\item Ian Drosos [C.33]
\item Logan Gittelson
\item Sean Kross [C.39]
\item Jaime Montoya
\item Xiong Zhang [C.36,C.43]
\end{innerlist}

\end{comment}

\begin{comment}

\section{Outreach@Scale}

My Python Tutor programming education website \url{pythontutor.com} has
attracted over 3.5 million total users from over 180 countries. My
personal website \url{pgbovine.net} contains over 300 articles, videos,
and podcast episodes on topics ranging from research to education, and
receives over 750,000 page views per year. I have also recorded over 500
videos on research, education, and outreach topics for my
YouTube channel, which now has 4,000+ subscribers and 500,000+
total video views.

\end{comment}




\begin{comment}
\ifx\public\undefined

\section{} \vspace{0.2in} \textbf{REFERENCES}

\vspace{0.5em}
Rob Miller \\
Professor \\
MIT EECS and MIT CSAIL \\
Massachusetts Institute of Technology \\
rcm@mit.edu

\vspace{1em}
Margo Seltzer \\
Herchel Smith Professor of Computer Science \\
Harvard John A. Paulson School of Engineering and Applied Sciences \\
Harvard University \\
margo@eecs.harvard.edu

\vspace{1em}
Jeffrey Heer \\
Associate Professor \\
Computer Science \& Engineering \\
University of Washington \\
jheer@uw.edu

\vspace{1em}
Dawson Engler \\
Associate Professor \\
Computer Science and Electrical Engineering \\
Stanford University \\
engler@stanford.edu

\fi
\end{comment}

\end{document}
